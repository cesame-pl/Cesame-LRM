% An array is an ordered set of primitive types, structures, functions and arrays.
Arrays in Cesame, like in many programming languages, are ordered collections of elements of the same type (both primitive and non-primitive). They allow random access to elements using indices and come with built-in methods for manipulating and accessing array elements. 

Furthermore, Cesame allows arrays to be nested, meaning you can have arrays of arrays.

\subsubsection{Declaring Arrays}
An array is declared using the keyword "Array" followed by arrow brackets to specify explicitly the type of contained elements. The following declarations are samples. Each declaration should end with a semicolon.
\begin{lstlisting}[caption={array\_declaration.csm}, captionpos=b]
struct Foo {
    int a;
    int b;
};
Array<int> a1;
Array<String> a2;
Array<Foo> a3;
Array<Array<int>> a4;
Array<Func<int,int>> a5;
\end{lstlisting}

\subsubsection{Defining Arrays}
An array is defined either explicitly using the "new" keyword followed by the "Array" keyword followed by arrow brackets ($<>$) indicating the type and followed by square brackets ($[\ \ ]$) containing the element separated by the colons. Every definition of an array ends with a semicolon. The following definition is an example.
\begin{lstlisting}[caption={array\_definition.csm}, captionpos=b]
Array<int> a1;
Array<int> a2;
a1 = new Array<int> [1, 2, 3];
a2 = new Array<int> [];

\end{lstlisting}
Note that the length of the array will be inferred exactly as the number of elements in the definition. For example, the example above has a length of 3 because there are three integers in the array. For an empty array, the length is 0.
\par A syntactic sugar for defining an array is also provided, which is just the square brackets containing the elements separated by columns ending with a semicolon. For example, the following definition is equivalent to the array definition in the last example.
\begin{lstlisting}[caption={array\_definition\_short.csm}, captionpos=b]
Array<int> a1;
Array<int> a2;
a1 = [1, 2, 3];
a2 = [];


\end{lstlisting}

The definition and declaration can be combined in one line.
\begin{lstlisting}[caption={array\_definition\_oneline.csm}, captionpos=b]
Array<int> a1 = [1, 2, 3];
Array<int> a2 = new Array<int> [];

\end{lstlisting}

\subsubsection{Array Member Functions}
The member functions are built-in functions that are not over-writable by users and can be called by an array. The function itself is just like a member in a struct. Suppose that we have already defined a struct T and an array of Array$<T>$ whose name is a, all the member functions are shown in the following table.
\begin{table}[h]
    \centering
    \setlength{\arrayrulewidth}{0.3mm}
    \renewcommand{\arraystretch}{1.2}
    \rowcolors{2}{gray!25}{white}
    \begin{tabular}{ccc}
        \toprule
        Member function & Equivalent Func object & Functionality \\
        \midrule
        a.len & Func$<$int$>$ & return the number of elements in the array \\
        \arrayrulecolor{gray!50}\cmidrule(lr){1-3}\arrayrulecolor{black}
        a.copy & Func$<$Array$<$T$>$$>$ & return a deep copy of the array \\
        \arrayrulecolor{gray!50}\cmidrule(lr){1-3}\arrayrulecolor{black}
        a.reverse & Func$<$Array$<$T$>$$>$ & return the reverse of the array \\
        \arrayrulecolor{gray!50}\cmidrule(lr){1-3}\arrayrulecolor{black}
        a.insert\_front & Func$<$T,void$>$ & in-place insertion before the first element \\
        \arrayrulecolor{gray!50}\cmidrule(lr){1-3}\arrayrulecolor{black}
        a.insert\_back  & Func$<$T,void$>$ & in-place insertion after the last element \\
        \arrayrulecolor{gray!50}\cmidrule(lr){1-3}\arrayrulecolor{black}
        a.insert & Func$<$int,T,void$>$ & in-place insertion before nth element \\
        \bottomrule
    \end{tabular}
\end{table}  \\

\par Calling these functions is just like calling a normal function, which is passing input in declared order in a pair of brackets right after the function. For example, a.insert\_front(m) is equivalent to a.insert(0,m). Note that if the array is declared but not defined, calling member functions will still pass the compilation but running the code will throw a runtime error.

\subsubsection{Indexing Arrays}
\par By providing the index, which is an integer between 0 (included) and the length of the array (not included), within a pair of square brackets right after the name of the array, the user can access any element in this array. The index $n$ means $(n+1)th$ element. Any out-of-bound indexing or indexing in an empty array will pass the compilation but lead to a runtime error. For example, the following function computes the sum of an integer array.
\begin{lstlisting}[caption={array\_sum.csm}, captionpos=b]
Func<Array<int>, int> sum = new Func (Array<int> a) -> int 
{
    int res = 0;
    for (int i = 0; i < a.len(); i++)
        res = res + a[i];
    return res;
}

\end{lstlisting}

\subsubsection{Deleting Arrays}
Deleting an array will free all the heap space occupied by this variable. It is done by using the "delete" keyword followed by the array variable name ending with a semicolon. For example, if we have already declared an array a, then we can delete it using
\begin{lstlisting}[caption={array\_delete.csm}, captionpos=b]
delete a;

\end{lstlisting}
If the array is declared but not defined, delete it will still pass the compilation but will throw a runtime error.

\subsubsection{Assigning Arrays}
Assigning an array to another array is supported using the "=" operator, with the array being assigned to the left-hand side, but the effect is assigning the reference of the right-hand-side array, which means that the two variable names point to the same array. For a defined array, assigning it to another array will cause a memory leak. If a deep copy of the right-hand-side array is desired, the copy member function should be used. Here are the examples of array assignments. Note that if the array is not defined, assigning another array to it will not be a definition but an assignment by reference.
\begin{lstlisting}[caption={array\_delete.csm}, captionpos=b]
Array<int> a = [1,2,3];
Array<int> b = a; /*Allowed, assign the reference of a to b*/
b[2] = 4; /*a = [1,2,4], b = [1,2,4]*/
Array<int> c = a.copy(); /*Deep copy, definition, C = [1,2,4]*/
delete b; /*a,b all freed, c = [1,2,4]*/

\end{lstlisting}