\documentclass[11pt,A4]{article}

\usepackage{listings}
\usepackage[table]{xcolor}
\usepackage{xcolor}
\usepackage {color}
\usepackage[margin=1.2in]{geometry}
\usepackage{parskip}
\usepackage{threeparttable}
\usepackage{graphicx}
\usepackage{amsmath, amsfonts, amsthm}
\usepackage{booktabs}

\newcommand{\horrule}[1]{\rule{\linewidth}{#1}} 
\newcommand{\todo}[1]{\textcolor{red}{#1}}

\definecolor{gray}{RGB}{211, 211, 211}
\definecolor{dkgreen}{rgb}{0,0.6,0}
\definecolor{mauve}{rgb}{0.58,0,0.82}

% \lstset{
%     language=C,
%     backgroundcolor=\color{gray!25!white},
%     showstringspaces=false,
%     breaklines=true,
%     basicstyle=\ttfamily,
%     xleftmargin=10mm,
%     xrightmargin=10mm,
%     framexleftmargin=10mm,
%     framexrightmargin=10mm,
% }

\lstset{frame=tb,
	language=c, % 使用的语言
	aboveskip=3mm,
	belowskip=3mm,
	showstringspaces=false, % 仅在字符串中允许空格
	backgroundcolor=\color{gray!25!white},   % 选择代码背景,必须加上\ usepackage {color}或\ usepackage {xcolor}
	columns=flexible,
	basicstyle = \ttfamily\small,
	% frame = shadowbox,  
	numbers=left, % 给代码添加行号,可取值none, left, right.
	numberstyle=\small \color{gray},  % 行号的字号和颜色
	keywordstyle=\color{blue},
	commentstyle=\color{dkgreen}, % 设置注释格式
	stringstyle=\color{mauve},
	breaklines=true,   % 设置自动断行.
	breakatwhitespace=true, % 设置是否当且仅当在空白处自动中断.
	escapeinside=``, %逃逸字符(1左面的键),用于显示中文
	% frame=shadowbox, %设置边框格式
	extendedchars=false, %解决代码跨页时,章节标题,页眉等汉字不显示的问题
	xleftmargin=2em,xrightmargin=2em, aboveskip=1em, %设置边距
	tabsize=4 % 将默认tab设置为4个空格
}

\title{
    \normalfont \LARGE
    \horrule{1pt} \\[0.4cm] 
    \huge The Cesame Programming Language \\
    \horrule{1pt} \\[0.6cm] 
    \textsc{Programming Language Reference Manual} \\ [25pt]
    % \Large IDK...Feel free to change (•‿•) \\[0.5cm]
}

\author{
   \begin{tabular}{ll}
       Language Guru: & Qian Zhao (qz2512) \\[5pt]
       System Architect: & Teng Jiang (tj2488) \\[5pt]
       Tester: & Yunjia Wang (yw4105)
   \end{tabular}
}
\date{}

\begin{document}

    \maketitle
    \thispagestyle{plain}
    \newpage
    
    \tableofcontents
    
    \newpage
    \section{Introduction}
    Your introduction content goes here.

    \newpage
    \section{Lexical Elements}
    
        \subsection{Identifiers}
        Identifiers are unique names given to various program elements such as variables, functions or arrays. Here are the naming rules:
        \begin{itemize}
            \item Case sensitive
            \item Begin with a letter or an underscore character
            \item Consist of only letters, digits, or underscore
            \item Cannot be a keyword
            \item Cannot contain white space
        \end{itemize}
        
        \subsection{Keywords}
        \todo{Fill Functions keywords in}
        \begin{center}
            \begin{tabular}{ c c c c c c c c }
            true & false \\
            new & delete \\ 
            void & bool & int & float & char & String & struct & Array \\
            if & else & for & while & goto & break & continue & return \\
            printf \\ 
            \end{tabular}
        \end{center}
        
        \subsection{Separators}
        A separator is used to separate tokens, including \texttt{white\_space} \texttt{( )} \texttt{\{ \}} \texttt{;} \texttt{,} \texttt{.} \texttt{:}

        \subsection{Operators}
        See the section "Expression and Operators"
        
        \subsection{White Space}
        White space in Cesame refers to characters that are used for formatting, such as spaces, tabs, and newline characters. White space is primarily used to enhance code readability and to separate tokens within a statement. 
        
        \subsection{Comment}
        Comments are non-executable lines used to document code and ignored by the compiler. There are two types of comments:
        \begin{itemize}
            \item Line comments
                \begin{lstlisting}
// line comments
                \end{lstlisting}
            \item General comments
                \begin{lstlisting}
/* General comments */
                \end{lstlisting}
        \end{itemize}

    \newpage
    \section{Data Types}
        \subsection{Primitive Data Types}
            \subsubsection{bool}
            \texttt{bool} can be either \texttt{true} or \texttt{false}. The logical operators can be applied to evaluate bool results.
            
            \subsubsection{int}
             \texttt{int} represents whole numbers, both positive and negative, without any fractional or decimal part. \texttt{int} occupies 4 bytes (32 bits) of memory, so its range is from -2,147,483,648 to 2,147,483,647.
            
            \subsubsection{float}
            \texttt{float} is a double-precision floating point number. It uses 8 bytes (64 bits) of memory and ranges from 1.7e-308 to 1.7e+308.
            
            \subsubsection{char}
            A character is a single printable ASCII character. It can be assigned and defined using a ASCII character or "escape sequences", such as \textbackslash n between a pair of single quotation marks. The supported escape sequences are the same as that of C. For the convenience of calculation, assigning a character variable to or defining it with a 1-byte unsigned integer is also supported and it is equivalent to the character at the corresponding position in ASCII table. Characters also supports "+" and "-" operators and they will behave as if 1-byte unsigned integers. In the memory the character will take one byte and stored as an unsigned integer. The followings are valid definitions, assignments and operations of characters.
\begin{lstlisting}[caption={char\_definition.csm}, captionpos=b]
char a = 'a';
char b = 'b';
char n = '\n';
b = a + 1;
b = 65; // b = 'A'
\end{lstlisting}
            
        \subsection{Strings}
        A string is an ordered sequence of characters. It can be created and deleted. Cesame has some member functions to help manipulate strings.
\subsubsection{Declaring Strings}
A string can be declared by the keyword "String" followed by a variable name ending with a semicolon. For example
\begin{lstlisting}[caption={string\_declaration.csm}, captionpos=b]
String s;
\end{lstlisting}
\subsubsection{Defining Strings}
A string can be defined with an assignment with a variable name on the left and a sequence of ASCII characters or "escaping sequences" between a pair of normal quotation marks ending with a semicolon. Definition and declaration in one line are supported. The new keyword is also supported by using a "new" keyword followed by a keyword "String" followed by a a pair of quotation marks between which are the ASCII characters.
\begin{lstlisting}[caption={string\_definition.csm}, captionpos=b]
String s = "Hello Cesame!\n";
String s1;
s1 = "s";
\end{lstlisting}
\subsubsection{String Member Functions}
Suppose that there is a defined string with variable name s, the member functions can be accessed by a dot following the variable name and called like a function. The member function is not over-writable by users. Calling the member function of an undefined string will throw a runtime error but will pass the compilation.
\begin{table}[h]
    \centering
    \setlength{\arrayrulewidth}{0.3mm}
    \renewcommand{\arraystretch}{1.2}
    \rowcolors{2}{gray!25}{white}
    \begin{tabular}{ccc}
        \toprule
        Member function & Equivalent Func object & Functionality \\
        \midrule
        s.len & Func$<$int$>$ & return the number of characters in the string \\
        \arrayrulecolor{gray!50}\cmidrule(lr){1-3}\arrayrulecolor{black}
        s.copy & Func$<$String$>$ & return a deep copy of the string \\
        
        \arrayrulecolor{gray!50}\cmidrule(lr){1-3}\arrayrulecolor{black}
        s.append & Func$<$String, String$>$ & return a string with the input appended right after s \\
        \arrayrulecolor{gray!50}\cmidrule(lr){1-3}\arrayrulecolor{black}
        s.print & Func$<$String, void$>$ & print s \\
        \bottomrule
    \end{tabular}
\end{table}  \\
\begin{lstlisting}[caption={string\_member\_function.csm}, captionpos=b]
String s = "Hello Cesame!\n";
s.len(); /*14*/
s.append(s); /*"Hello Cesame!\nHello Cesame!\n"*/
\end{lstlisting}
\subsubsection{Indexing Strings}
Indexing a string will return the reference of the character in the corresponding position. The index can be done by following the variable name with a pair of square brackets and an integer n indicating the position between the square brackets with $n \in [0, len(s))$ ending with a semicolon. An out-of-bound indexing will throw a runtime error but will still pass the compilation. Note that indexing is not totally equivalent to Func$<$int,char$>$ because what it returns is the reference, which means using an index can change the original string.
\begin{lstlisting}[caption={indexing\_string.csm}, captionpos=b]
String s = "Bad";
s[1] = 'e'; // s = "Bed"
\end{lstlisting}
\subsubsection{Assigning Strings}
Assigning a string to another string will do the reference assignment, which means that these two variables will point to the same string. If a new string is needed, the "copy" member function is provided to make a deep copy. Assigning a string to a defined string will cause a memory leak if the defined string is not deleted.
\begin{lstlisting}[caption={assigning\_string.csm}, captionpos=b]
String s = "Bad";
String s1 = s;
String s2 = s.copy();
s1[1] = 'e'; // s = "Bed", s1 = "Bed", s2 = "Bad"
\end{lstlisting}
\subsubsection{Deleting Strings}
A string variable needs to be deleted in order to free the heap space occupied by it once being defined. It can be done by using the "delete" keyword followed by the variable name ending with a semicolon. Freeing an undefined string will cause a runtime error but will pass the compilation.
\begin{lstlisting}[caption={deleting\_string.csm}, captionpos=b]
String s = "Bad";
delete s;
\end{lstlisting}

        
        \subsection{Structures}
        A structure is an unordered collection of different variables, including primitive types, String, structure and arrays. It can be declared, defined, and instantiated.
\subsubsection{Declaring Structures}
The structure is declared using the keyword "struct" followed by the structure name ending with a semicolon. The structure name should follow the naming convention of a non-primitive type, which means that the first letter should be capitalized. The following example declares a structure called Student.
\begin{lstlisting}[caption={struct\_declaration.csm}, captionpos=b]
    struct Student;
\end{lstlisting}
Note that at this time the structure is not defined and cannot be instantiated.
\begin{lstlisting}[caption={struct\_not\_defined.csm}, captionpos=b]
    struct Student;
    Student a; /*Throw a runtime error*/
\end{lstlisting}
\subsubsection{Defining Structures}
The structure can be defined in three ways. One is using assignment, the left-hand-side is the structure name and the right-hand-side is the "new" keyword followed by the struct keyword followed by a pair of curly brackets (\{\}) ending with a semicolon. The members in the structure should be declared within the curly brackets. For example, the structure Student can be defined as
\begin{lstlisting}[caption={struct\_definition.csm}, captionpos=b]
    struct Student;
    Student = new struct {
        String name;
        int age;
    };
\end{lstlisting}
Another abbreviated way is not to use new struct keywords on the right-hand side.
\begin{lstlisting}[caption={struct\_definition.csm}, captionpos=b]
    struct Student;
    Student = {
        String name;
        int age;
    };
\end{lstlisting}
To make the structure definition more like that of C, the assignment can also be omitted.
\begin{lstlisting}[caption={struct\_definition.csm}, captionpos=b]
    struct Student;
    Student {
        String name;
        int age;
    };
\end{lstlisting}
Putting definitions and declarations together is also allowed.
\begin{lstlisting}[caption={struct\_definition.csm}, captionpos=b]
    struct Student {String name; int age;};
    struct Teacher = new struct {String name; int age};
    struct Alien = {String from;};
\end{lstlisting}

\subsubsection{Instantiating Structures}
Defining a structure is like defining a non-primitive type, declaring a variable of such a type is called instantiating the structure and the variable is called an instance. Instantiating the structure is just a declaration, the members of the structure are still not defined. The members of the structure can be accessed by structure name followed by a dot followed by the declared member name and the members can be defined one by one. Also, a convenient way of defining the members is provided for defining the variables after or when instantiating, which is following the instance name with a pair of curly brackets and the definitions in the declared order.
\begin{lstlisting}[caption={struct\_instantiate.csm}, captionpos=b]
    struct Student {String name; int age;};
    Student stu;
    Student stu1 {name = "Qian"; age = -5;};
    Student stu2 {name = "Ronghui"; age = 0;};
    stu.name = "abc";
    stu.age = 12345;
\end{lstlisting}
\subsubsection{Deleting Instances of Structures}
Using delete followed by the instance name ending with a semicolon will free the heap space of the instance. Forgetting to manually delete the instance will cause a memory leak.
\begin{lstlisting}[caption={struct\_delete.csm}, captionpos=b]
    struct Student {String name; int age;};
    Student stu1 {name = "Qian"; age = -5;};
    delete stu1;
\end{lstlisting}
\subsubsection{Assigning Instances of Structures}
Instances of structures supports the assignment operator, with the reference of right-hand side assigned to the right-hand side, meaning they are pointing to the same instance after the assignment. Deep copy function should be implemented manually.
\begin{lstlisting}[caption={struct\_deep\_cpoy.csm}, captionpos=b]
    struct Student stu {String name; int age;};
    Func<Student, Student, void> deep_copy_stu = new Func (Student a, Student b) -> void {
        b.name = a.name.copy();
        b.age = a.age;
    };
    Student stu1 {name = "who"; age = -1;};
    Student stu2 = stu1;
    stu2.age = 7; /*stu1.age = 7*/
    Student stu3;
    deep_copy_stu(stu1, stu3);
    stu3.age = 8; /*stu1.age = 7*/
\end{lstlisting}



        
        \subsection{Arrays}
        An array is a ordered set of primitive types, structures functions and array.
\subsubsection{Declaring Arrays}
An array is declared using keyword "Array" followed by arrow brackets to specify explicitly the type of contained elements. The following declarations are samples. Each declaration should end with a semicolon.
\begin{lstlisting}[caption={array\_declaration.csm}, captionpos=b]
struct Foo {
    int a;
    int b;
};
Array<int> a1;
Array<String> a2;
Array<Foo> a3;
Array<Array<int>> a4;
Array<Func<int,int>> a5;
\end{lstlisting}

\subsubsection{Defining Arrays}
An array is defined either explicitly using "new" keyword followed by "Array" keyword followed by arrow brackets ($<>$) indicating the type and followed by square bracket ($[\ \ ]$) containing the element separated by the colons. Every definition of an array ends with a semicolon. The following definition is an example.
\begin{lstlisting}[caption={array\_definition.csm}, captionpos=b]

Array<int> a1;
Array<int> a2;
a1 = new Array<int> [1, 2, 3];
a2 = new Array<int> [];

\end{lstlisting}
Note that the length of the array will be inferred exactly as the number of elements in the definition. For example, the example above has a length of 3 because there are three integers in the array. For empty array, the length is 0.
\par A shortcut of defining an array is also provided, which is just the square brackets containing the elements separated by columns ending with a semicolon. For example, the following definition is equivalent to the array definition in the last example.
\begin{lstlisting}[caption={array\_definition\_short.csm}, captionpos=b]

Array<int> a1;
Array<int> a2;
a1 = [1, 2, 3];
a2 = [];


\end{lstlisting}

The definition and declaration can be combined in one line.
\begin{lstlisting}[caption={array\_definition\_oneline.csm}, captionpos=b]

Array<int> a1 = [1, 2, 3];
Array<int> a2 = new Array<int> [];

\end{lstlisting}

\subsubsection{Array Member Functions}
The member functions are built-in functions that is not over-writable by users and can be called by an array. The function itself is just like a member in a struct. Suppose that we have already defined a struct T and an array of Array$<T>$ whose name is a, all the member functions are shown in the following table.
\begin{table}[h]
\centering
\begin{tabular}{|c|c|c|}
\hline
Member function & Equivalent Func object & Functionality                               \\ \hline
a.len           & Func$<$int$>$          & return the number of elements in the array  \\ \hline
a.copy          & Func$<$Array$<$T$>$$>$ & return a deep copy of the array             \\ \hline
a.reverse       & Func$<$Array$<$T$>$$>$ & return the reverse of the array             \\ \hline
a.insert\_front & Func$<$T,void$>$       & in-place insertion before the first element \\ \hline
a.insert\_back  & Func$<$T,void$>$       & in-place insertion after the last element   \\ \hline
a.insert        & Func$<$int,T,void$>$   & in-place insertion before nth element       \\ \hline
\end{tabular}
\end{table}
\par Calling these functions is just like calling a normal function, which is passing input in declared order in a pair of brackets right after the function. For example, a.insert\_front(m) is equivalent to a.insert(0,m). Note that if the array is declared but not defined, calling member functions will still pass the compilation but running the code will throw a runtime error.

\subsubsection{Indexing Arrays}
\par By providing the index, which is an integer between 0 (included) and the length of the array (not included), within a pair of square brackets right after the name of the array, the user can access any element in this array. The index $n$ means $(n+1)th$ element. Any out-of-bound indexing or indexing in an empty array will pass the compilation but lead to a runtime error. For example, the following function computes the sum of an integer array.
\begin{lstlisting}[caption={array\_sum.csm}, captionpos=b]

Func<Array<int>, int> sum = new Func (Array<int> a) -> int 
{
    int res = 0;
    for (int i = 0; i < a.len(); i++)
        res = res + a[i];
    return res;
}

\end{lstlisting}

    \newpage
    \section{Expressions and Operators}
        
        \subsection{Operands}
        Operands are entities upon which operators act within expressions. They can be values, variables, constants, or the results of sub-expressions. For example, in the expression a + b, a and b are operands.
        
        \subsection{Unary Operators}
        Unary operators are operators in programming that operate on a single operand, meaning they work with only one operand or value. \\
        \begin{table}[h]
            \centering
            \setlength{\arrayrulewidth}{0.3mm}
            \renewcommand{\arraystretch}{1.2}
            \rowcolors{2}{gray!25}{white}
            \begin{tabular}{cccc}
                \toprule
                Symbol & Operator & Description & Example \\
                \midrule
                ++ & Increment & Increases the value of a variable by 1 & 
                \begin{tabular}[t]{@{}c@{}} 
                    \verb|int x = 5;| \\
                    \verb|x++;|
                \end{tabular} \\
                \arrayrulecolor{gray!50}\cmidrule(lr){1-4}\arrayrulecolor{black}
                - - & Decrement & Decreases the value of a variable by 1 & 
                \begin{tabular}[t]{@{}c@{}} 
                    \verb|int y = 10;| \\
                    \verb|y--;|
                \end{tabular} \\
                \arrayrulecolor{gray!50}\cmidrule(lr){1-4}\arrayrulecolor{black}
                + & Unary plus & Indicates a positive value & 
                \begin{tabular}[t]{@{}c@{}} 
                    \verb|int z = -5;| \\
                    \verb|int posZ = +z;|
                \end{tabular} \\
                \arrayrulecolor{gray!50}\cmidrule(lr){1-4}\arrayrulecolor{black}
                - & Unary minus & Negates the value of an expression & 
                \begin{tabular}[t]{@{}c@{}} 
                    \verb|int a = 8;| \\
                    \verb|int negA = -a;|
                \end{tabular} \\
                \arrayrulecolor{gray!50}\cmidrule(lr){1-4}\arrayrulecolor{black}
                ! & Logical NOT & Performs logical negation & 
                \begin{tabular}[t]{@{}c@{}} 
                    \verb|int isTrue = 1;| \\
                    \verb|int isFalse = !isTrue;|
                \end{tabular} \\
                \bottomrule
            \end{tabular}
        \end{table}  \\ 
        
        \subsection{Binary Operators}
        Binary operators are operators in programming that require two operands to perform an operation. 
            \newpage
            \subsubsection{Arithmetic Operators} 
            Perform mathematical calculations on numeric values.
            \begin{table}[h]
                \centering
                \setlength{\arrayrulewidth}{0.3mm}
                \renewcommand{\arraystretch}{1.2}
                \rowcolors{2}{gray!25}{white}
                \begin{tabular}{cccc}
                    \toprule
                    Symbol & Operator & Description & Example \\
                    \midrule
                    + & Addition & Adds two operands & \verb|int x1 = 5 + 3;| \\
                    \arrayrulecolor{gray!50}\cmidrule(lr){1-4}\arrayrulecolor{black}
                    - & Subtraction & Subtracts second operand from the first & \verb|int x2 = 7 - 2;| \\
                    \arrayrulecolor{gray!50}\cmidrule(lr){1-4}\arrayrulecolor{black}
                    * & Multiplication & Multiplies two operands & \verb|int x3 = 4 * 6;| \\
                    \arrayrulecolor{gray!50}\cmidrule(lr){1-4}\arrayrulecolor{black}
                    / & Division & Divides first operand by the second & \verb|float x4 = 10.0 / 2.0;| \\
                    \arrayrulecolor{gray!50}\cmidrule(lr){1-4}\arrayrulecolor{black}
                    \% & Modulus & Returns the remainder of the division & \verb|int x5 = 10 % 3;| \\
                    \bottomrule
                \end{tabular}
            \end{table} 
            
            \subsubsection{Comparison Operators}
            Compare two values and determine the relationship between them.
            \begin{table}[h]
                \centering
                \setlength{\arrayrulewidth}{0.3mm}
                \renewcommand{\arraystretch}{1.2}
                \rowcolors{2}{gray!25}{white}
                \begin{tabular}{ccc}
                    \toprule
                    Symbol & Operator & Description \\
                    \midrule
                    == & Equal to & Checks if two operands are equal \\
                    \arrayrulecolor{gray!50}\cmidrule(lr){1-3}\arrayrulecolor{black}
                    != & Not equal to & Checks if two operands are not equal \\
                    \arrayrulecolor{gray!50}\cmidrule(lr){1-3}\arrayrulecolor{black}
                    > & Greater than & Checks if first operand is greater than second \\
                    \arrayrulecolor{gray!50}\cmidrule(lr){1-3}\arrayrulecolor{black}
                    < & Less than & Checks if first operand is less than second \\
                    \arrayrulecolor{gray!50}\cmidrule(lr){1-3}\arrayrulecolor{black}
                    >= & Greater than or equal to & Checks if first operand is greater than or equal to second \\
                    \arrayrulecolor{gray!50}\cmidrule(lr){1-3}\arrayrulecolor{black}
                    <= & Less than or equal to & Checks if first operand is less than or equal to second \\
                    \bottomrule
                \end{tabular}
            \end{table}  
                        
            \subsubsection{Logical Operators} 
            Perform logical operations on boolean values.
            \begin{table}[h]
                \centering
                \setlength{\arrayrulewidth}{0.3mm}
                \renewcommand{\arraystretch}{1.2}
                \rowcolors{2}{gray!25}{white}
                \begin{tabular}{cccc}
                    \toprule
                    Symbol & Operator & Description & Example \\
                    \midrule
                    \&\& & Logical AND & Returns true if both operands are true & \verb|a && b| \\
                    \arrayrulecolor{gray!50}\cmidrule(lr){1-4}\arrayrulecolor{black}
                    || & Logical OR & Returns true if either operand is true & \verb|x| || \verb|y| \\
                    \arrayrulecolor{gray!50}\cmidrule(lr){1-4}\arrayrulecolor{black}
                    ! & Logical NOT & Inverts the truth value of its operand & \verb|!a| \\
                    \bottomrule
                \end{tabular}
            \end{table} 
            
            \subsubsection{Bitwise Operators}
            Perform operations at the bit level.
            \begin{table}[h]
                \centering
                \setlength{\arrayrulewidth}{0.3mm}
                \renewcommand{\arraystretch}{1.2}
                \rowcolors{2}{gray!25}{white}
                \begin{tabular}{cccc}
                    \toprule
                    Symbol & Operator & Description & Example \\
                    \midrule
                    \& & Bitwise AND & Performs bitwise AND on each pair of bits & \verb|a & b| \\
                    \arrayrulecolor{gray!50}\cmidrule(lr){1-4}\arrayrulecolor{black}
                    | & Bitwise OR & Performs bitwise OR on each pair of bits & \verb|x| | \verb|y| \\
                    \arrayrulecolor{gray!50}\cmidrule(lr){1-4}\arrayrulecolor{black}
                    $\wedge$ & Bitwise XOR & Performs bitwise XOR on each pair of bits & \verb|p ^ q| \\
                    \arrayrulecolor{gray!50}\cmidrule(lr){1-4}\arrayrulecolor{black}
                    $\sim$ & Bitwise NOT & Inverts all the bits & \verb|~a;| \\
                    \arrayrulecolor{gray!50}\cmidrule(lr){1-4}\arrayrulecolor{black}
                    $<<$ & Left shift & Shifts bits to the left & \verb|x << 2| \\
                    \arrayrulecolor{gray!50}\cmidrule(lr){1-4}\arrayrulecolor{black}
                    $>>$ & Right shift & Shifts bits to the right & \verb|y >> 3| \\
                    \bottomrule
                \end{tabular}
            \end{table} 
            
            \subsubsection{Assignment Operators}
            Assign values to variables.
            \begin{table}[h]
                \centering
                \setlength{\arrayrulewidth}{0.3mm}
                \renewcommand{\arraystretch}{1.2}
                \rowcolors{2}{gray!25}{white}
                \begin{tabular}{ccc}
                    \toprule
                    Symbol & Operator & Example \\
                    \midrule
                    = & Assignment & \verb|int x = 5;| \\
                    \arrayrulecolor{gray!50}\cmidrule(lr){1-3}\arrayrulecolor{black}
                    += & Addition assignment & \verb|x += 3;| \\
                    \arrayrulecolor{gray!50}\cmidrule(lr){1-3}\arrayrulecolor{black}
                    -= & Subtraction assignment & \verb|x -= 2;| \\
                    \arrayrulecolor{gray!50}\cmidrule(lr){1-3}\arrayrulecolor{black}
                    *= & Multiplication assignment & \verb|x *= 4;| \\
                    \arrayrulecolor{gray!50}\cmidrule(lr){1-3}\arrayrulecolor{black}
                    /= & Division assignment & \verb|x /= 2;| \\
                    \arrayrulecolor{gray!50}\cmidrule(lr){1-3}\arrayrulecolor{black}
                    \%= & Modulus assignment & \verb|x \% = 3;| \\
                    \arrayrulecolor{gray!50}\cmidrule(lr){1-3}\arrayrulecolor{black}
                    <<= & Left shift assignment & \verb|x <<= 2;| \\
                    \arrayrulecolor{gray!50}\cmidrule(lr){1-3}\arrayrulecolor{black}
                    >>= & Right shift assignment & \verb|x >>= 3;| \\
                    \arrayrulecolor{gray!50}\cmidrule(lr){1-3}\arrayrulecolor{black}
                    \&= & Bitwise AND assignment & \verb|x \&= y;| \\
                    \arrayrulecolor{gray!50}\cmidrule(lr){1-3}\arrayrulecolor{black}
                    |= & Bitwise OR assignment & \verb|x| |= \verb|y;| \\
                    \arrayrulecolor{gray!50}\cmidrule(lr){1-3}\arrayrulecolor{black}
                    $\wedge$= & Bitwise XOR assignment & \verb|x ^= y;| \\
                    \bottomrule
                \end{tabular}
            \end{table}

        \subsection{Ternary Operators}
        \begin{lstlisting}
        condition ? expression1 : expression2;
        \end{lstlisting}
        The condition is a boolean expression that results in either true or false. If the condition is 
        \begin{itemize}
            \item \textbf{true}: expression1 is executed
            \item \textbf{false}: expression2 is executed
        \end{itemize}

        \subsection{Expressions}
        An expression is a combination of values, variables, operators, and function calls that evaluates to a single value. It can be as simple as a single variable or value, or it can be complex, involving multiple operations. \\

        Expressions may include:
        \begin{itemize}
            \item \textbf{Literals}: Constants such as numbers, characters, or strings.
                \begin{itemize}
                    \item \texttt{int num = 10;}
                    \item \texttt{char ch = 'A';}
                    \item \texttt{String s = "Hello";}
                \end{itemize}
                
            \item \textbf{Variables}: Named storage locations that hold values.
                \begin{itemize}
                    \item \texttt{int a = 5;}
                    \item \texttt{float b = 3.14;}
                    \item \texttt{char ch = 'B';}
                \end{itemize}
                
            \item \textbf{Operators}: Symbols representing computations such as addition, subtraction, division, etc.
                \begin{itemize}
                    \item \texttt{int sum = 5 + 3;} 
                    \item \texttt{int diff = 7 - 2;}
                    \item \texttt{float div = 10.0 / 2.0;}
                \end{itemize}
                
            \item \textbf{Function Calls}: Invocations of functions that return values.
                \begin{itemize}
                    \item \texttt{int minVal = min(10, 20);}
                    \item \texttt{int absValue = abs(-5);}
                    \item \texttt{float sqrtValue = sqrt(25.0);}
                \end{itemize}
        \end{itemize}

        \subsection{Operator Precedence and Associativity}
        The operators are grouped according to the rules of precedence when an expression contains multiple operators. The following list shows types of expressions presented in order of highest precedence first. For the operators of the same precedence, its associativity is left to right unless stated otherwise.
        \begin{enumerate}
            \item Array subscription, function calls, and membership accesses
            \item Unary operators. When multiple unary operators are consecutive, the later ones are nested in the earlier ones: !-a means !(-a)
            \item Multiplication, division, and modular expressions
            \item Addition and subtraction expressions
            \item Bitwise shifting expressions
            \item >=, <=, >, < expressions
            \item ==, != expressions
            \item Bitwise AND expressions
            \item Bitwise XOR expressions
            \item Bitwise OR expressions
            \item Logical AND expressions
            \item Logical OR expressions
            \item Ternary expressions
            \item Assignment expressions. Multiple assignment statements are evaluated from right to left. 
        \end{enumerate}

    \newpage
    \section{Statements}
        \subsection{Blocks}
        A block statement is a group of possibly empty statements enclosed in curly braces. Blocks serve several important purposes including grouping statements, scoping and variable lifetime, conditional execution, and code isolation.
        \begin{lstlisting}
Func print_val() -> int {
    // Block begins
    {
        int a = 5;
        printf("Value of a: %d\n", a);
    }
    // Block ends

    return 0;
}
        \end{lstlisting}
        
        \subsection{The Empty Statements}
        An empty statement does not do anything and is merely a semicolon alone. Mostly, an empty statement is used as the body of a loop statement. It is also used to follow a label that would otherwise be the last statement in a block. 
        \begin{lstlisting}
        for (int i = 0; i < 10; i++)
            ;
        \end{lstlisting}
        
        \subsection{Labeled Statement}
        Labels are identifiers followed by a colon, used to mark a position within the code. These labeled positions can be targeted by control flow statements such as `goto`, `break`, and `continue`. A labeled statement is a statement followed by a label.\\ \\
        Here's the general syntax for a labeled statement:
        \begin{lstlisting}
        label_name: statement
        \end{lstlisting}
        
        \subsection{Expression Statements}
        An expression statement is a statement that consists of an expression. Expression statements are only useful for their side effects, such as modifying variables, calling functions, or performing input/output operations.
        \begin{lstlisting}
// Function call is an expression statement
printf("Hello, world!\n");
        \end{lstlisting}

        \subsection{Declaration statements}
        A declaration statement introduces a new identifier (variable or function name) to the program. It specifies the type of the identifier and optionally initializes it with a value.
        \begin{lstlisting}
int x; // Declaration of variable x of type int
        \end{lstlisting}

        \subsection{Assignment statements}
        An assignment statement assigns a value to an already declared variable.
        \begin{lstlisting}
x = 10; // Assignment of the value 10 to the variable x
int y = x + 5;  // Declaration and assignment combined
        \end{lstlisting}
        
        \subsection{The \texttt{if} Statement}
        The \texttt{if} statement is a flow control statement used for executing a block of code conditionally.
        \begin{itemize}
            \item \textbf{\texttt{if} Statement}
            \begin{lstlisting}
if (condition) {
   // code to be executed if the condition is true
}
            \end{lstlisting}
            
            \item \textbf{\texttt{if-else} Statement}
            \begin{lstlisting}
if (condition) {
    // code executed when the condition is true
} else {
    // code executed when the condition is false
}
            \end{lstlisting}

            \item \textbf{\texttt{if-else-if} Ladder}
            \begin{lstlisting}
if (condition1) {
    // code executed when  condition1 is true
} else if (condition2) { 
    // code executed when condition1 is false and condition2 is true
} else {
    // code executed when both condition1 and condition2 are false
}
            \end{lstlisting}
        \end{itemize}
        
        \subsection{The \texttt{while} Statement}
        The \texttt{while} statement is a control flow statement that repeatedly executes a block of code as long as a specified condition is true.
        \begin{lstlisting}
while (condition) {
    // Code block to be executed repeatedly
}
        \end{lstlisting}
        
        \subsection{The \texttt{for} Statement}
        The \texttt{for} statement is a versatile control flow statement used for iterative execution. It allows you to execute a block of code repeatedly, with precise control over the iteration process. 
        \begin{lstlisting}
for (initialization; condition; step) {
    // Code block to be executed repeatedly
}
        \end{lstlisting}
        
        \subsection{The \texttt{goto} Statement}
        The \texttt{goto} statement is a control flow statement that allows you to transfer program control to a labeled statement within the same function.
        \begin{lstlisting}
goto label;
        \end{lstlisting}
    
        \subsection{The \texttt{break} Statement}
        the \texttt{break} statement is a control flow statement used to terminate the execution of a loop or switch statement prematurely.
        \begin{lstlisting}
break;
        \end{lstlisting}
        
        \subsection{The \texttt{continue} Statement}
        The \texttt{continue} statement is a control flow statement used to skip the current iteration of a loop and proceed to the next iteration. 
        \begin{lstlisting}
continue;
        \end{lstlisting}
        
        \subsection{The \texttt{return} Statement}
        The \texttt{return} statement is used to terminate the execution of a function and return a value to the calling code.
        \begin{lstlisting}
return <expression>;
        \end{lstlisting}

    \newpage
    \section{Functions}
        Functions are considered "first-class citizens" in Cesame, which means that a function can be assigned to variables, passed as arguments to other functions, and returned from other functions. Just like other non-primitive types in Cesame, we can declare and define a function object. However, Cesame employs a different syntax for function definition compared to C, reflecting its unique design principles.

        \subsection{Function Declaration}
            Functions in Cesame accept zero or more parameters as inputs and return zero or one output. A named function is characterized by the function name and the types of its inputs and output. An anonymous function is only characterized by the types of its inputs and output.


The types of inputs and output are specified in the bracket $<>$ followed by "Func", and separated with ",". There has to be at least one element in the bracket. The last element should be either a type specifying the output type, or the void keyword indicating that we output type. "void" keywords that are not the last element are ignored.

        \begin{lstlisting}
/* f1 takes in 2 integers and returns an integer. */
Func<int, int, int> f1;

/* f2_1 takes in nothing and returns an integer. */
Func<int> f2_1; 

/* void ignored, equivalent with f2_1 */
Func<void, int> f2_2;

/* f3 takes in an integer and returns nothing */
Func<int, void> f3; 

/* Functions can both be inputs and outputs and also can take in and return non-primitive types. */
Func<Func<String, Array<int>>, Func<String>> f4;
        \end{lstlisting}
        
        \subsection{Function Definitions}

An anonymous function is defined using the following syntax.

\begin{lstlisting}
/* An anonymous function that increments an integer */
(int x)->int { return x + 1 }; 

/* An anonymous function prints a string but does not return anything */
(String s)->void { s.print(); }

/* An anonymous function that does not take in anything, just returns 42 */
(String s)->int { return 42; }
\end{lstlisting}

There are 3 ways to define a function: 

Firstly, by using the "new" keyword we provide just like what we do with another non-primitive type. Notice that because the type is going to be able to be inferred from the definition, we can just use "new Func" instead of "new Func<...>", because the type is specified later in the body of the definition. This is the first kind of type inference allowed in functions.

\begin{lstlisting}
Func<int, int> inc;
inc = new Func (int i)-> int { return i + 1; };
\end{lstlisting}


If you declare and define a function together, the input and output types can also be inferred.

\begin{lstlisting}
Func inc = new Func (int i)-> int { return i + 1; };
\end{lstlisting}




Secondly, by using the "function" keyword (a syntactic sugar) we provide, you can declare and define a function together, similar to the syntax in C.
\begin{lstlisting}
function inc (int i)-> int { return i + 1; };
Func<int, int> inc (int i)-> int { return i + 1; };
\end{lstlisting}

Thirdly, you can define a function with a defined function, using "=", with LHS being the undefined function and RHS being a defined function. Notice that here "=" is definition, not assignment.
\begin{lstlisting}
/* Function declared, but not defined */
Func<int, int> inc, dec;

function another_dec (int i)-> int { return i - 1; }

/* inc is defined with a lambda function */
inc = (int i)->int { return i + 1 }; 

/* dec is defined with a named function another_dec */
dec = another_dec;


\end{lstlisting}



% Define an anonymous
% Define a named function: We can assign an unnamed function to it, or use the syntactic sugar we provided.
% Do we allow assigning a named function to a named function? (reference cnt += 1?)
% We also provide a syntactic sugar that. Here, the type of function f can be inferred, from the type of the anonymous function.

        
        
\subsection{Calling Functions}
Calling a function or in the sense of first-class functions evaluating a function, has a similar syntax with C.
\begin{lstlisting}
Func inc = new Func (int i)->int { return i + 1; };
int a = inc(a);
\end{lstlisting}

\subsection{Assign Functions}
Lastly, a defined function (named or anonymous) can be assigned to a defined named function.

\begin{lstlisting}
Func inc = new Func (int i)->int { return i + 1; };
Func another_inc = new Func (int i)->int { return i + 1; };

/* assign a named function to inc */
inc = another_inc;

/* assign a lambda function to inc */
inc = (int i)->int { return i + 1; };

\end{lstlisting}
        
        % \subsection{Function Parameters}
        % Content for function parameters.
        
        % \subsection{Higher-Order Functions}
        % Content for higher-order functions.

    \newpage
    \section{Sample Program}
    Content for sample program.

    \newpage
    \section{References}
    Content for reference.

\subsection{Assigning Functions}
Lambda function can not be assigned to. Ass
    
\end{document}
