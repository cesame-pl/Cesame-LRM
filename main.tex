\documentclass[11pt,A4]{article}

\usepackage{listings}
\usepackage[table]{xcolor}
\usepackage{xcolor}
\usepackage[margin=1.2in]{geometry}
\usepackage{parskip}
\usepackage{threeparttable}
\usepackage{graphicx}
\usepackage{amsmath,amsfonts,amsthm}
\usepackage{booktabs}


\newcommand{\horrule}[1]{\rule{\linewidth}{#1}} 
\newcommand{\todo}[1]{\textcolor{red}{#1}}

\definecolor{gray}{RGB}{211, 211, 211}

\lstset{
    language=C,
    backgroundcolor=\color{gray!25!white},
    showstringspaces=false,
    breaklines=true,
    basicstyle=\ttfamily,
    xleftmargin=10mm,  
    xrightmargin=10mm,
    framexleftmargin=10mm,
    framexrightmargin=10mm,
}

\title{
    \normalfont \LARGE
    \horrule{1pt} \\[0.4cm] 
    \huge The Cesame Programming Language \\
    \horrule{1pt} \\[0.6cm] 
    \textsc{Programming Language Reference Manual} \\ [25pt]
    % \Large IDK...Feel free to change (•‿•) \\[0.5cm]
}

\author{
   \begin{tabular}{ll}
       Language Guru: & Qian Zhao (qz2512) \\[5pt]
       System Architect: & Teng Jiang (tj2488) \\[5pt]
       Tester: & Yunjia Wang (yw4105)
   \end{tabular}
}
\date{}

\begin{document}

    \maketitle
    \thispagestyle{plain}
    \newpage
    
    \tableofcontents
    
    \newpage
    \section{Introduction}
    Your introduction content goes here.

    \newpage
    \section{Lexical Elements}
    
        \subsection{Identifiers}
        Content for identifiers.
        
        \subsection{Keywords}
        Content for keywords.
        
        \subsection{Constants}
            \subsubsection{Integer Constants}
            Content for integer constants.
            
            \subsubsection{Character Constants}
            Content for character constants.
        
            \subsubsection{Real Number Constants}
            Content for real number constants.
            
            \subsubsection{String Constants}
            Content for string constants.
        
        \subsection{Separators}
        Content for separators.
        
        \subsection{Operators}
        Content for operators.
        
        \subsection{White Space}
        Content for white space.
        
        \subsection{Comment}
        Content for comment.

    \newpage
    \section{Data Types}
        \subsection{Primitive Data Types}
            \subsubsection{bool}
            Content for bool.
            
            \subsubsection{int}
            Content for int.
            
            \subsubsection{float}
            Content for float.
            
            \subsubsection{char}
            Content for char.
            
        \subsection{Strings}
        Content for strings.
        
        \subsection{Enumerations}
        Content for enumerations.
        
        \subsection{Unions}
        Content for unions.
        
        \subsection{Structures}
        Content for structures.
        
        \subsection{Arrays}
        % An array is an ordered set of primitive types, structures, functions and arrays.
Arrays in Cesame, like in many programming languages, are ordered collections of elements of the same type (both primitive and non-primitive). They allow random access to elements using indices and come with built-in methods for manipulating and accessing array elements. 

Furthermore, Cesame allows arrays to be nested, meaning you can have arrays of arrays.

\subsubsection{Declaring Arrays}
An array is declared using the keyword "Array" followed by arrow brackets to specify explicitly the type of contained elements. The following declarations are samples. Each declaration should end with a semicolon.
\begin{lstlisting}[caption={array\_declaration.csm}, captionpos=b]
struct Foo {
    int a;
    int b;
};
Array<int> a1;
Array<String> a2;
Array<Foo> a3;
Array<Array<int>> a4;
Array<Func<int,int>> a5;
\end{lstlisting}

\subsubsection{Defining Arrays}
An array is defined either explicitly using the "new" keyword followed by the "Array" keyword followed by arrow brackets ($<>$) indicating the type and followed by square brackets ($[\ \ ]$) containing the element separated by the colons. Every definition of an array ends with a semicolon. The following definition is an example.
\begin{lstlisting}[caption={array\_definition.csm}, captionpos=b]
Array<int> a1;
Array<int> a2;
a1 = new Array<int> [1, 2, 3];
a2 = new Array<int> [];

\end{lstlisting}
Note that the length of the array will be inferred exactly as the number of elements in the definition. For example, the example above has a length of 3 because there are three integers in the array. For an empty array, the length is 0.
\par A syntactic sugar for defining an array is also provided, which is just the square brackets containing the elements separated by columns ending with a semicolon. For example, the following definition is equivalent to the array definition in the last example.
\begin{lstlisting}[caption={array\_definition\_short.csm}, captionpos=b]
Array<int> a1;
Array<int> a2;
a1 = [1, 2, 3];
a2 = [];


\end{lstlisting}

The definition and declaration can be combined in one line.
\begin{lstlisting}[caption={array\_definition\_oneline.csm}, captionpos=b]
Array<int> a1 = [1, 2, 3];
Array<int> a2 = new Array<int> [];

\end{lstlisting}

\subsubsection{Array Member Functions}
The member functions are built-in functions that are not over-writable by users and can be called by an array. The function itself is just like a member in a struct. Suppose that we have already defined a struct T and an array of Array$<T>$ whose name is a, all the member functions are shown in the following table.
\begin{table}[h]
    \centering
    \setlength{\arrayrulewidth}{0.3mm}
    \renewcommand{\arraystretch}{1.2}
    \rowcolors{2}{gray!25}{white}
    \begin{tabular}{ccc}
        \toprule
        Member function & Equivalent Func object & Functionality \\
        \midrule
        a.len & Func$<$int$>$ & return the number of elements in the array \\
        \arrayrulecolor{gray!50}\cmidrule(lr){1-3}\arrayrulecolor{black}
        a.copy & Func$<$Array$<$T$>$$>$ & return a deep copy of the array \\
        \arrayrulecolor{gray!50}\cmidrule(lr){1-3}\arrayrulecolor{black}
        a.reverse & Func$<$Array$<$T$>$$>$ & return the reverse of the array \\
        \arrayrulecolor{gray!50}\cmidrule(lr){1-3}\arrayrulecolor{black}
        a.insert\_front & Func$<$T,void$>$ & in-place insertion before the first element \\
        \arrayrulecolor{gray!50}\cmidrule(lr){1-3}\arrayrulecolor{black}
        a.insert\_back  & Func$<$T,void$>$ & in-place insertion after the last element \\
        \arrayrulecolor{gray!50}\cmidrule(lr){1-3}\arrayrulecolor{black}
        a.insert & Func$<$int,T,void$>$ & in-place insertion before nth element \\
        \bottomrule
    \end{tabular}
\end{table}  \\

\par Calling these functions is just like calling a normal function, which is passing input in declared order in a pair of brackets right after the function. For example, a.insert\_front(m) is equivalent to a.insert(0,m). Note that if the array is declared but not defined, calling member functions will still pass the compilation but running the code will throw a runtime error.

\subsubsection{Indexing Arrays}
\par By providing the index, which is an integer between 0 (included) and the length of the array (not included), within a pair of square brackets right after the name of the array, the user can access any element in this array. The index $n$ means $(n+1)th$ element. Any out-of-bound indexing or indexing in an empty array will pass the compilation but lead to a runtime error. For example, the following function computes the sum of an integer array.
\begin{lstlisting}[caption={array\_sum.csm}, captionpos=b]
Func<Array<int>, int> sum = new Func (Array<int> a) -> int 
{
    int res = 0;
    for (int i = 0; i < a.len(); i++)
        res = res + a[i];
    return res;
}

\end{lstlisting}

\subsubsection{Deleting Arrays}
Deleting an array will free all the heap space occupied by this variable. It is done by using the "delete" keyword followed by the array variable name ending with a semicolon. For example, if we have already declared an array a, then we can delete it using
\begin{lstlisting}[caption={array\_delete.csm}, captionpos=b]
delete a;

\end{lstlisting}
If the array is declared but not defined, delete it will still pass the compilation but will throw a runtime error.

\subsubsection{Assigning Arrays}
Assigning an array to another array is supported using the "=" operator, with the array being assigned to the left-hand side, but the effect is assigning the reference of the right-hand-side array, which means that the two variable names point to the same array. For a defined array, assigning it to another array will cause a memory leak. If a deep copy of the right-hand-side array is desired, the copy member function should be used. Here are the examples of array assignments. Note that if the array is not defined, assigning another array to it will not be a definition but an assignment by reference.
\begin{lstlisting}[caption={array\_delete.csm}, captionpos=b]
Array<int> a = [1,2,3];
Array<int> b = a; /*Allowed, assign the reference of a to b*/
b[2] = 4; /*a = [1,2,4], b = [1,2,4]*/
Array<int> c = a.copy(); /*Deep copy, definition, C = [1,2,4]*/
delete b; /*a,b all freed, c = [1,2,4]*/

\end{lstlisting}

    \newpage
    \section{Expressions and Operators}
        
        \subsection{Operands}
        Operands are the data items or entities upon which operators act within expressions. They can be variables, constants, or the results of sub-expressions. Operands are the values or variables involved in an operation. For example, in the expression a + b, a and b are operands.
        
        \subsection{Unary Operators}
        Unary operators are operators in programming that operate on a single operand, meaning they work with only one operand or value. \\
        \todo{TODO: do we still have pointers? remove the Dereference if not.}
        \begin{table}[h]
            \centering
            \setlength{\arrayrulewidth}{0.3mm}
            \renewcommand{\arraystretch}{1.2}
            \rowcolors{2}{gray!25}{white}
            \begin{tabular}{cccc}
                \toprule
                Symbol & Operator & Description & Example \\
                \midrule
                ++ & Increment & Increases the value of a variable by 1 & 
                \begin{tabular}[t]{@{}c@{}} 
                    \verb|int x = 5;| \\
                    \verb|x++;|
                \end{tabular} \\
                \arrayrulecolor{gray!50}\cmidrule(lr){1-4}\arrayrulecolor{black}
                -- & Decrement & Decreases the value of a variable by 1 & 
                \begin{tabular}[t]{@{}c@{}} 
                    \verb|int y = 10;| \\
                    \verb|y--;|
                \end{tabular} \\
                \arrayrulecolor{gray!50}\cmidrule(lr){1-4}\arrayrulecolor{black}
                + & Unary plus & Indicates a positive value & 
                \begin{tabular}[t]{@{}c@{}} 
                    \verb|int z = -5;| \\
                    \verb|int positiveZ = +z;|
                \end{tabular} \\
                \arrayrulecolor{gray!50}\cmidrule(lr){1-4}\arrayrulecolor{black}
                - & Unary minus & Negates the value of an expression & 
                \begin{tabular}[t]{@{}c@{}} 
                    \verb|int a = 8;| \\
                    \verb|int b = -a;|
                \end{tabular} \\
                \arrayrulecolor{gray!50}\cmidrule(lr){1-4}\arrayrulecolor{black}
                ! & Logical NOT & Performs logical negation & 
                \begin{tabular}[t]{@{}c@{}} 
                    \verb|int isTrue = 0;| \\
                    \verb|int isFalse = !isTrue;|
                \end{tabular} \\
                \arrayrulecolor{gray!50}\cmidrule(lr){1-4}\arrayrulecolor{black}
                \& & Address-of & Returns the memory address of a variable & 
                \begin{tabular}[t]{@{}c@{}} 
                    \verb|int var = 42;| \\
                    \verb|int* ptr = &var;|
                \end{tabular} \\
                \arrayrulecolor{gray!50}\cmidrule(lr){1-4}\arrayrulecolor{black}
                * & Dereference & Accesses the value pointed to by a pointer & 
                \begin{tabular}[t]{@{}c@{}} 
                    \verb|int var = 42;| \\
                    \verb|int* ptr = &var;| \\
                    \verb|int value = *ptr;|
                \end{tabular} \\
                \bottomrule
            \end{tabular}
        \end{table}  \\ 
    
        \subsection{Binary Operators}
        Binary operators are operators in programming that require two operands to perform an operation. 
        
            \subsubsection{Arithmetic Operators} 
            Perform mathematical calculations on numeric values
            \begin{table}[h]
                \centering
                \setlength{\arrayrulewidth}{0.3mm}
                \renewcommand{\arraystretch}{1.2}
                \rowcolors{2}{gray!25}{white}
                \begin{tabular}{cccc}
                    \toprule
                    Symbol & Operator & Description & Example \\
                    \midrule
                    + & Addition & Adds two operands & \verb|int sum = 5 + 3;| \\
                    \arrayrulecolor{gray!50}\cmidrule(lr){1-4}\arrayrulecolor{black}
                    - & Subtraction & Subtracts second operand from the first & \verb|int difference = 7 - 2;| \\
                    \arrayrulecolor{gray!50}\cmidrule(lr){1-4}\arrayrulecolor{black}
                    * & Multiplication & Multiplies two operands & \verb|int product = 4 * 6;| \\
                    \arrayrulecolor{gray!50}\cmidrule(lr){1-4}\arrayrulecolor{black}
                    / & Division & Divides first operand by the second & \verb|float quotient = 10.0 / 2.0;| \\
                    \arrayrulecolor{gray!50}\cmidrule(lr){1-4}\arrayrulecolor{black}
                    \% & Modulus & Returns the remainder of the division & \verb|int remainder = 10 % 3;| \\
                    \bottomrule
                \end{tabular}
            \end{table}  \\ 
            
            \subsubsection{Relational Operators}
            Compare two values and determine the relationship between them
            \begin{table}[h]
                \centering
                \setlength{\arrayrulewidth}{0.3mm}
                \renewcommand{\arraystretch}{1.2}
                \rowcolors{2}{gray!25}{white}
                \begin{tabular}{ccc}
                    \toprule
                    Symbol & Operator & Description \\
                    \midrule
                    == & Equal to & Checks if two operands are equal \\
                    \arrayrulecolor{gray!50}\cmidrule(lr){1-3}\arrayrulecolor{black}
                    != & Not equal to & Checks if two operands are not equal \\
                    \arrayrulecolor{gray!50}\cmidrule(lr){1-3}\arrayrulecolor{black}
                    > & Greater than & Checks if first operand is greater than second \\
                    \arrayrulecolor{gray!50}\cmidrule(lr){1-3}\arrayrulecolor{black}
                    < & Less than & Checks if first operand is less than second \\
                    \arrayrulecolor{gray!50}\cmidrule(lr){1-3}\arrayrulecolor{black}
                    >= & Greater than or equal to & Checks if first operand is greater than or equal to second \\
                    \arrayrulecolor{gray!50}\cmidrule(lr){1-3}\arrayrulecolor{black}
                    <= & Less than or equal to & Checks if first operand is less than or equal to second \\
                    \bottomrule
                \end{tabular}
            \end{table}  \\ 
                        
            \subsubsection{Logical Operators} 
            perform logical operations on boolean values
            \begin{table}[h]
                \centering
                \setlength{\arrayrulewidth}{0.3mm}
                \renewcommand{\arraystretch}{1.2}
                \rowcolors{2}{gray!25}{white}
                \begin{tabular}{cccc}
                    \toprule
                    Symbol & Operator & Description & Example \\
                    \midrule
                    \&\& & Logical AND & Returns true if both operands are true & \verb|a && b| \\
                    \arrayrulecolor{gray!50}\cmidrule(lr){1-4}\arrayrulecolor{black}
                    || & Logical OR & Returns true if either operand is true & \verb|x| || \verb|y| \\
                    \arrayrulecolor{gray!50}\cmidrule(lr){1-4}\arrayrulecolor{black}
                    ! & Logical NOT & Inverts the truth value of its operand & \verb|!a| \\
                    \bottomrule
                \end{tabular}
            \end{table}  \\ 
            
            \subsubsection{Bitwise Operators}
            Perform operations at the bit level
            \begin{table}[h]
                \centering
                \setlength{\arrayrulewidth}{0.3mm}
                \renewcommand{\arraystretch}{1.2}
                \rowcolors{2}{gray!25}{white}
                \begin{tabular}{cccc}
                    \toprule
                    Symbol & Operator & Description & Example \\
                    \midrule
                    \& & Bitwise AND & Performs bitwise AND on each pair of bits & \verb|a & b| \\
                    \arrayrulecolor{gray!50}\cmidrule(lr){1-4}\arrayrulecolor{black}
                    | & Bitwise OR & Performs bitwise OR on each pair of bits & \verb|x| | \verb|y| \\
                    \arrayrulecolor{gray!50}\cmidrule(lr){1-4}\arrayrulecolor{black}
                    $\wedge$ & Bitwise XOR & Performs bitwise XOR on each pair of bits & \verb|p ^ q| \\
                    \arrayrulecolor{gray!50}\cmidrule(lr){1-4}\arrayrulecolor{black}
                    $\sim$ & Bitwise NOT & Inverts all the bits & \verb|~a;| \\
                    \arrayrulecolor{gray!50}\cmidrule(lr){1-4}\arrayrulecolor{black}
                    << & Left shift & Shifts bits to the left & \verb|x << 2| \\
                    \arrayrulecolor{gray!50}\cmidrule(lr){1-4}\arrayrulecolor{black}
                    >> & Right shift & Shifts bits to the right & \verb|y >> 3| \\
                    \bottomrule
                \end{tabular}
            \end{table}  \\ 
            
            \subsubsection{Assignment Operators}
            Assign values to variables
            \begin{table}[h]
                \centering
                \setlength{\arrayrulewidth}{0.3mm}
                \renewcommand{\arraystretch}{1.2}
                \rowcolors{2}{gray!25}{white}
                \begin{tabular}{ccc}
                    \toprule
                    Symbol & Operator & Example \\
                    \midrule
                    = & Assignment & \verb|int x = 5;| \\
                    \arrayrulecolor{gray!50}\cmidrule(lr){1-3}\arrayrulecolor{black}
                    += & Addition assignment & \verb|x += 3;| \\
                    \arrayrulecolor{gray!50}\cmidrule(lr){1-3}\arrayrulecolor{black}
                    -= & Subtraction assignment & \verb|x -= 2;| \\
                    \arrayrulecolor{gray!50}\cmidrule(lr){1-3}\arrayrulecolor{black}
                    *= & Multiplication assignment & \verb|x *= 4;| \\
                    \arrayrulecolor{gray!50}\cmidrule(lr){1-3}\arrayrulecolor{black}
                    /= & Division assignment & \verb|x /= 2;| \\
                    \arrayrulecolor{gray!50}\cmidrule(lr){1-3}\arrayrulecolor{black}
                    \%= & Modulus assignment & \verb|x \% = 3;| \\
                    \arrayrulecolor{gray!50}\cmidrule(lr){1-3}\arrayrulecolor{black}
                    <<= & Left shift assignment & \verb|x <<= 2;| \\
                    \arrayrulecolor{gray!50}\cmidrule(lr){1-3}\arrayrulecolor{black}
                    >>= & Right shift assignment & \verb|x >>= 3;| \\
                    \arrayrulecolor{gray!50}\cmidrule(lr){1-3}\arrayrulecolor{black}
                    \&= & Bitwise AND assignment & \verb|x \&= y;| \\
                    \arrayrulecolor{gray!50}\cmidrule(lr){1-3}\arrayrulecolor{black}
                    |= & Bitwise OR assignment & \verb|x |= y;| \\
                    \arrayrulecolor{gray!50}\cmidrule(lr){1-3}\arrayrulecolor{black}
                    $\wedge$= & Bitwise XOR assignment & \verb|x ^= y;| \\
                    \bottomrule
                \end{tabular}
            \end{table}

        \subsection{Ternary Operators}
        \begin{lstlisting}
        condition ? expression1 : expression2;
        \end{lstlisting}
        The condition is a boolean expression that results in either true or false. If the condition is 
        \begin{itemize}
            \item \textbf{true}: expression1 is executed
            \item \textbf{false}: expression2 is executed
        \end{itemize}
        
        \subsection{Operator Precedence and Associativity}
        \todo{TODO: Content for operator precedence and associativity.}

        \subsection{Expressions}
        An expression is a combination of values, variables, operators, and function calls that evaluates to a single value. It can be as simple as a single variable or value, or it can be complex, involving multiple operations. \\
        \todo{TODO: revisit the syntax once finalized.}

        Expressions may include:
        \begin{itemize}
            \item \textbf{Literals}: Constants such as numbers, characters, or strings.
                \begin{itemize}
                    \item \texttt{int num = 10;}
                    \item \texttt{char ch = 'A';}
                    \item \texttt{char str[] = "Hello";}
                \end{itemize}
                
            \item \textbf{Variables}: Named storage locations that hold values.
                \begin{itemize}
                    \item \texttt{int a = 5;}
                    \item \texttt{float b = 3.14;}
                    \item \texttt{char ch = 'B';}
                \end{itemize}
                
            \item \textbf{Operators}: Symbols representing computations such as addition, subtraction, division, etc.
                \begin{itemize}
                    \item \texttt{int sum = 5 + 3;} 
                    \item \texttt{int diff = 7 - 2;}
                    \item \texttt{float div = 10.0 / 2.0;}
                \end{itemize}
                
            \item \textbf{Function Calls}: Invocations of functions that return values.
                \begin{itemize}
                    \item \texttt{int minVal = min(10, 20);}
                    \item \texttt{int absValue = abs(-5);}
                    \item \texttt{float sqrtValue = sqrt(25.0);}
                    
                \end{itemize}
        \end{itemize}

    \newpage
    \section{Statements} 
        \todo{TODO: revisit the syntax once finalized.}
        \subsection{Blocks}
        A block statement is a group of possibly empty statements enclosed in curly braces. Blocks serve several important purposes including grouping statements, scoping and variable lifetime, conditional execution, and code isolation.
        \begin{lstlisting}
int main() {
    // Block begins
    {
        int a = 5;
        printf("Value of a: %d\n", a);
    }
    // Block ends

    return 0;
}
        \end{lstlisting}
        
        \subsection{The Empty Statements}
        An empty statement does not do anything and is merely a semicolon alone. Mostly, an empty statement is used as the body of a loop statement. It is also used to follow a label that would otherwise be the last statement in a block. 
        \begin{lstlisting}
        for (int i = 0; i < 10; i++)
            ;
        \end{lstlisting}
        
        \subsection{Labeled Statement}
        Labels are identifiers followed by a colon, used to mark a position within the code. These labeled positions can be targeted by control flow statements such as `goto`, `break`, and `continue`. A labeled statement is a statement followed by a label.\\ \\
        Here's the general syntax for a labeled statement:
        \begin{lstlisting}
        label_name: statement
        \end{lstlisting}
        
        \subsection{Expression Statements}
        An expression statement is a statement that consists of an expression. Expression statements are only useful for their side effects, such as modifying variables, calling functions, or performing input/output operations.
        \begin{lstlisting}
// Function call is an expression statement
printf("Hello, world!\n");
        \end{lstlisting}

        \subsection{Declaration statements}
        A declaration statement introduces a new identifier (variable or function name) to the program. It specifies the type of the identifier and optionally initializes it with a value.
        \begin{lstlisting}
int x; // Declaration of variable x of type int
        \end{lstlisting}

        \subsection{Assignment statements}
        An assignment statement assigns a value to an already declared variable.
        \begin{lstlisting}
x = 10; // Assignment of the value 10 to the variable x
int y = x + 5;  // Declaration and assignment combined
        \end{lstlisting}
        
        \subsection{The \texttt{if} Statement}
        The \texttt{if} statement is a flow control statement used for executing a block of code conditionally.
        \begin{itemize}
            \item \textbf{\texttt{if} Statement}
            \begin{lstlisting}
if (condition) {
   // code to be executed if the condition is true
}
            \end{lstlisting}
            
            \item \textbf{\texttt{if-else} Statement}
            \begin{lstlisting}
if (condition) {
    // code executed when the condition is true
} else {
    // code executed when the condition is false
}
            \end{lstlisting}

            \item \textbf{\texttt{if-else-if} Ladder}
            \begin{lstlisting}
if (condition1) {
    // code executed when  condition1 is true
} else if (condition2) { 
    // code executed when condition1 is false and condition2 is true
} else {
    // code executed when both condition1 and condition2 are false
}
            \end{lstlisting}
        \end{itemize}
        
        \subsection{The \texttt{switch} Statement}
        \todo{TODO: do we need switch?}
        
        \subsection{The \texttt{while} Statement}
        The \texttt{while} statement is a control flow statement that repeatedly executes a block of code as long as a specified condition is true.
        \begin{lstlisting}
while (condition) {
    // Code block to be executed repeatedly
}
        \end{lstlisting}
        
        \subsection{The \texttt{do} Statement}
        \todo{TODO: \texttt{do} we need do?}
        
        \subsection{The \texttt{for} Statement}
        The \texttt{for} statement is a versatile control flow statement used for iterative execution. It allows you to execute a block of code repeatedly, with precise control over the iteration process. 
        \begin{lstlisting}
for (initialization; condition; step) {
    // Code block to be executed repeatedly
}
        \end{lstlisting}
        
        \subsection{The \texttt{goto} Statement}
        The \texttt{goto} statement is a control flow statement that allows you to transfer program control to a labeled statement within the same function.
        \begin{lstlisting}
goto label;
        \end{lstlisting}
    
        \subsection{The \texttt{break} Statement}
        the \texttt{break} statement is a control flow statement used to terminate the execution of a loop or switch statement prematurely.
        \begin{lstlisting}
break;
        \end{lstlisting}
        
        \subsection{The \texttt{continue} Statement}
        The \texttt{continue} statement is a control flow statement used to skip the current iteration of a loop and proceed to the next iteration. 
        \begin{lstlisting}
continue;
        \end{lstlisting}
        
        \subsection{The \texttt{return} Statement}
        The \texttt{return} statement is used to terminate the execution of a function and return a value to the calling code.
        \begin{lstlisting}
return <expression>;
        \end{lstlisting}
        
        \subsection{The \texttt{typedef} Statement}
        The \texttt{typedef} statement is used to create new data type names (aliases) for existing data types, making code more readable and manageable.
        \begin{lstlisting}
typedef <existing_data_type> <new_data_type>;
        \end{lstlisting}

    \newpage
    \section{Functions}
        \subsection{Function Declaration}
        Functions are one of the mo
        Content for function declarations.
        
        \subsection{Function Definitions}
        Content for function definitions.
        
        \subsection{Calling Functions}
        Content for calling functions.
        
        \subsection{Function Parameters}
        Content for function parameters.
        
        \subsection{Higher-Order Functions}
        Content for higher-order functions.

    \newpage
    \section{Sample Program}
    Content for sample program.

    \newpage
    \section{References}
    Content for reference.
    
\end{document}
