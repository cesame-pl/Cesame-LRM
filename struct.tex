A structure is an unordered collection of different variables, including primitive types, String, structure and arrays. It can be declared, defined, and instantiated.
\subsubsection{Declaring Structures}
The structure is declared using the keyword "struct" followed by the structure name followed by member declarations between a pair of curly brackets ending with a semicolon. The structure name should follow the naming convention of a non-primitive type, which means that the first letter should be capitalized. The following example declares a structure called Student.
\begin{lstlisting}[caption={struct\_declaration.csm}, captionpos=b]
    struct Student {
        String name;
        int age;
    };
\end{lstlisting}

\subsubsection{Defining and Declaring Variables of Structures}
The structure can be defined in two ways. One is using assignment, the left-hand-side is the structure name followed by the variable name and the right-hand-side is the "new" keyword followed by the structure name followed by a pair of curly brackets (\{\}) ending with a semicolon. The members in the structure should be declared within the curly brackets. For example, the structure Student can be defined as
\begin{lstlisting}[caption={struct\_definition.csm}, captionpos=b]
    Student a = new Student {
        name = "Qian";
        age = -5;
    };
\end{lstlisting}
%Another abbreviated way is not to use new on the right-hand side.
%\begin{lstlisting}[caption={struct\_definition.csm}, captionpos=b]
%    Student a = Student {
%        name = "Qian";
%        age = -5;
%    };
%\end{lstlisting}
To make the structure definition more like that of C, the assignment can also be omitted.
\begin{lstlisting}[caption={struct\_definition.csm}, captionpos=b]
    Student a {
        name = "Qian";
        age = -5;
    };
\end{lstlisting}
Also, the variables of structures can be declared and but undefined.
\begin{lstlisting}[caption={struct\_definition.csm}, captionpos=b]
    Student a;
    a {name = "Qian", age = -5};
\end{lstlisting}
If a keyword "new" is used, the members can be undefined in the definition of the structure variable. Otherwise the members should be defined at the same time. The member can be accessed and defined by dot operation.
\begin{lstlisting}[caption={struct\_definition.csm}, captionpos=b]
    Student a, b, c;
    a = new Student; // Correct
    b {}; // Wrong, compilation error
    a.name = "Qian";
    a.age = -5;
\end{lstlisting}

\subsubsection{Deleting Structures}
Using delete followed by the instance name ending with a semicolon will free the heap space of the instance. Forgetting to manually delete the instance will cause a memory leak.
\begin{lstlisting}[caption={struct\_delete.csm}, captionpos=b]
    struct Student {String name; int age;};
    Student stu1 {name = "Qian"; age = -5;};
    delete stu1;
\end{lstlisting}
\subsubsection{Assigning Structures}
Instances of structures supports the assignment operator, with the reference of right-hand side assigned to the right-hand side, meaning they are pointing to the same instance after the assignment. Deep copy function should be implemented manually.
\begin{lstlisting}[caption={struct\_deep\_cpoy.csm}, captionpos=b]
    struct Student {String name; int age;};
    Func<Student, Student, void> deep_copy_stu = new Func (Student a, Student b) -> void {
        b.name = a.name.copy();
        b.age = a.age;
    };
    Student stu1 {name = "who"; age = -1;};
    Student stu2 = stu1;
    stu2.age = 7; /*stu1.age = 7*/
    Student stu3;
    deep_copy_stu(stu1, stu3);
    stu3.age = 8; /*stu1.age = 7*/
\end{lstlisting}


